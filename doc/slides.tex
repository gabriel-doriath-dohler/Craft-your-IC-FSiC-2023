\documentclass{beamer}

\title{Learning hardware design in the video game Minecraft\footnote{\tiny{}\reproduce}\\\small{} École Normale Supérieure Paris (ENS Ulm)}
\author{Gabriel \textsc{Doriath Döhler}, Constantin \textsc{Gierczak--Galle}}
\date{\paramDate\ifx\paramDate\empty\today\fi}

\usepackage{fontspec}
\setmonofont{FreeMono}
\usepackage{amsmath}
\usepackage{listings}
\usepackage{xcolor}

\lstset{
  language=haskell,
  escapeinside={&}{&},
  columns=fixed,
  extendedchars,
  basewidth={0.5em,0.45em},
  basicstyle=\ttfamily,
  mathescape,
}

\begin{document}
\beamertemplatenavigationsymbolsempty
\addtobeamertemplate{navigation symbols}{}{%
  \usebeamerfont{footline}%
  \usebeamercolor[fg]{footline}%
  \hspace{1em}%
  \insertframenumber/\inserttotalframenumber
}

\maketitle

\begin{frame}[fragile]
\frametitle{Motivation}

\end{frame}

\begin{frame}[fragile]
\frametitle{Table of Contents}
\tableofcontents
\end{frame}

\section{Introduction}

\begin{frame}[fragile]
\frametitle{Questions?}
\end{frame}

\newpage
\bibliographystyle{plain}
\bibliography{ref}

\end{document}
