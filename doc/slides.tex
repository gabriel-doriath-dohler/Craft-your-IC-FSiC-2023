%! TEX TS-program = lualatex
%! TeX root = doc/slides.tex

\documentclass[aspectratio=169]{beamer}

\newcommand{\names}{Gabriel \textsc{Doriath Döhler}, Constantin \textsc{Gierczak--Galle}}

\title{"Craft your IC"\\Learning hardware design in
Minecraft\footnote{\tiny{}\reproduce}\\\small{} École Normale Supérieure, Paris (ENS Ulm)}
\author{\names}
\institute{FSiC $2023$}
\date{July $12$, $2023$}

\usepackage{fontspec}
\setmonofont{FreeMono}
\usepackage{amsmath}
\usepackage{listings}
\usepackage{xcolor}
\usepackage{qrcode}
\usepackage{caption}

\newcommand{\rv}{\texttt{RISC-V}}
\newcommand{\vrv}{\texttt{V-\rv}}
\newcommand{\hw}{HW}
% FIXME: Update this with the new link
% Also, add a clickable link for the handouts
\newcommand{\rvlink}{https://github.com/gabriel-doriath-dohler/V-RISC-V}

\setcounter{tocdepth}{1}

\begin{document}
\beamertemplatenavigationsymbolsempty
\addtobeamertemplate{navigation symbols}{}{%
	\usebeamerfont{footline}%
	\usebeamercolor[fg]{footline}%
	\hspace{1em}%
	\insertframenumber/\inserttotalframenumber
}

\maketitle

\begin{frame}[fragile]
	\frametitle{Story time}
	game of life turing machine
	borrow computational construct from the game
	sysnum CPU, cst1 + gdd idée MC
	but there is one problem: know that we had the idea, we had to do it
	but why???
	- Minecraft is a good interactive tool to learn hardwar econcepts
	- BUT IT'S ALSO VERY FUN
\end{frame}

\begin{frame}[fragile]
	\frametitle{Table of Contents}
	\tableofcontents
\end{frame}

\section{Logic circuits in Minecraft (gdd)}

\begin{frame}
	\frametitle{Minecraft}
	block game owned by microsoft (not open source but reverse friendly)
	a player moves and can place/delete cubes (called blocks) in a grid
\end{frame}

\begin{frame}
	\frametitle{Minecraft Redstone}
	how redstone works

	wool
	lamps and levers = inputs and outputs
	redstone wire, or gates (FANIN), FANOUT
	torches and ticks (1 tick = .1 sec)
	simple torch based clock (in game)
	repeaters
	delay element
	diode
	amplifier
	can't have 4 meters of cables like in tiny tapeout
	locked repeaters = memory cell
	comparators and barrels
	slabs = instant diodes
	(many more constructs)

	minecraft as a SLOW weird netlist simulator
\end{frame}

\begin{frame}
	\frametitle{Building Circuits!}
	we already built a clock and a memory cell
	example of simple circuits
	half adder (in game)
	MUX/DEMUX?
	cool minecraft tech: CCA adder (Yap7 + Magic:\^ (2014)))
	$$cc_i = carry_cancel_i = cancel_i = a_i nand b_i = not propagate$$
	$$g_i = generate_i = a_1 and b_i$$
	$$c_{i+1} = (not cancel_i) and (generate_i or ((not cancel_{i-1}) and (generate_{i-1} or ...)))$$
	$$= ((not cancel_i) and generate_i) or ((not cancel_i) and (not cancel_{i-1}) and generate_{i-1}) or ...$$
	$$= sum_{j=0}^{i} generate_{j} and prod_{k=j}^{i} ¬cancel_k$$
	$$slabTower_{g_i} = max power strength_{j=0}^{i-1} generate_j$$
	$$slabTower_{cc_i} = max power strength_{j=0}^{i-1} cc_j$$
	works because a carry that is later canceled will have less power strength on th towers
	$$c_{i+1} = slabTower_{g_i} > slabTower_{cc_i}$$
	$$c_0 = carry_in$$
	$$a_n = b_n = 0$$
	$$o_i = a_i xor b_i xor c_i$$
	cancel doesn't have to be accurate (use NOR)

	7 ticks design (5 tick possible?)
	fully synchronized
	pipeline every 1 or 2 ticks
\end{frame}

\begin{frame}
	\frametitle{Minecraft vs "Real" Hardware: what's different?}
	power strength
	instant diodes
	no tooling: HDL, version control, GTKWave, test framework
	the physical layer is much more constraining: wires are big and connect automatically, but being compact is necessary for performance
\end{frame}

\section{"Craft your IC" Design and ISA (cst1)}

% note for cst1: i would talk about:
% - ISA choice (why not RISV-V?)
% - debuging (replay mod)
% - wirering the different parts is a challenge
% - decoding instruction is trivial because of instruction length
% - simple pipeline (wave pipeline) (not done yet)
% - the things you already mention

\begin{frame}
	\frametitle{Instruction Set Architecture}
	%\vrv
	%\texttt{(Very-RISC)-V}

	$8$-bit registers:

	\begin{itemize}
		\item \texttt{\%0}: always $00000000$ (from the \rv{} spec)
		\item \texttt{\%1}: always $11111111$ (Gives \texttt{inc},
		      \texttt{dec}, \texttt{not})
		\item \texttt{\%2}-\texttt{\%14}: GP registers
		\item \texttt{\%15}: Random register
	\end{itemize}

	\pause

	Instructions:
	\begin{itemize}
		\item \textbf{Binops}: \texttt{add}, \texttt{sub}, \texttt{or},
		      \texttt{xor}: $3$ register operands
		\item \textbf{Jumps}: inconditionnal or on previous operation
		      \texttt{oveflow}/\texttt{negative}/\texttt{zero}
		\item \texttt{loadi}: Load immediate to register
		\item \textbf{RAM}: Load and store
	\end{itemize}
\end{frame}

\subsection{An $8$-bit CPU (???)}
% Harvard architecthture (separate ROM and RAM)
% 8 bits because max power strength is 15 so the ALU towers do not need repeaters
\begin{frame}
	Rapidly list all components

	For each component, make a 2-column slide, the left column contains some
	interesting bulletpoints, the right one a picture of said components (if
	possible isolated?)
\end{frame}

\begin{frame}
	\frametitle{ROM and PC}
	\begin{columns}
		\begin{column}{0.5\textwidth}
			\begin{itemize}
				\item ??? number of instructions?
				\item How are the instructions stored?
			\end{itemize}
		\end{column}
		\begin{column}{0.5\textwidth}
			\begin{figure}
				An Image
				\caption*{The ALU}
			\end{figure}
		\end{column}
	\end{columns}
\end{frame}

\begin{frame}
	\frametitle{Registers}
	\begin{columns}
		\begin{column}{0.5\textwidth}
			\begin{itemize}
				\item All registers doubled for double--read
				\item Formal verification
			\end{itemize}
		\end{column}
		\begin{column}{0.5\textwidth}
			\begin{figure}
				An Image
				\caption*{Register file}
			\end{figure}
		\end{column}
	\end{columns}
\end{frame}

\begin{frame}
	\frametitle{I/O}
	\begin{columns}
		\begin{column}{0.5\textwidth}
			\begin{itemize}
				\item \hw{} design in MC
				\item Formal verification
			\end{itemize}
		\end{column}
		\begin{column}{0.5\textwidth}
			\begin{figure}
				An Image
				\caption*{I/O: screen}
			\end{figure}
		\end{column}
	\end{columns}
\end{frame}

\begin{frame}
	\frametitle{ALU}
	\begin{columns}
		\begin{column}{0.5\textwidth}
			\begin{itemize}
				\item Combinatorial
				\item Compact design for speed.
			\end{itemize}
		\end{column}
		\begin{column}{0.5\textwidth}
			\begin{figure}
				An Image
				\caption*{The ALU}
			\end{figure}
		\end{column}
	\end{columns}
\end{frame}

\section{Toolchain (cst1)}
% merge this section somewhere else?
\begin{frame}[fragile]
	\frametitle{Toolchain}
	\begin{itemize}
		\item Assembler
	\end{itemize}
	% sorry it's broken I should fix it
	%	\begin{lstlisting}
	%            \# thing
	%            store @12 \%5 \#test
	%            add \%0 \%2 \%30
	%            jmp abcsd
	%            or \%0 \%15 \%1
	%            loadi \$1 \%17
	%            sub \%0 \%0 \%5
	%            nop
	%        .issou
	%            mov \%2 \%5
	%            jz issou \# Jump
	%  \end{lstlisting}
\end{frame}

\section{Demo (cst1)}

\begin{frame}
	\frametitle{The Demo (finally!)}
	clock
	e approximation using RNG
	joke: very secure because: active temper detection with TNT + industry grade (aka. java) RNG.
\end{frame}

\section{The community (cst1)}

\begin{frame}
	ORE
	Chungus 2
	minecraft in minecraft
	similar to traditional FOSS communities
	license
	they make sure to give credits to creators of designs

	toolchain:
	- prismlauncher and quilt mod loader
	- ressource pack vanilla tweaks
	- worldedit (copy paste)
	- replay mod (debugging and rendering)
	- carpet (tick rate control + noclip and more)
	- ferritecore, immediatelyFast, lithium, sodium (optimizations)
	- Xaero's World Map (waypoints)
	- Tweakeroo (flexible and accurate block placement)
	- litematica
	- isometric renders (renders)
	- MCHPRS (gdd?)

	Talk about the community doing crazy CPUs in MC, drop a picture, some
	links, etc.
\end{frame}

\section{Perspectives and Verification (gdd can skip if needed)}
\begin{frame}
	\frametitle{Perspectives}
	\begin{itemize}
		\item arm cortex A53
		\item RV32-I
		\item dynamic FPGA
		\item MCHPS
		\item semantics and formal verification
	\end{itemize}
\end{frame}

\begin{frame}
	\frametitle{Verifying HW semantics}
	MC semantics
	ALU graph
	haskell k induction
	results
	stats number of blocks (part's list)
	so way too big for SMT
	plan: use model checking with a lean DSL and state machines



	Notes: On n'a pas de HDL (expliquer pourquoi et donner le lien du HDL existant)
	Mentionner l'état de l'art (~inexistant) en génération de circuit MC à
	partir de HDL.

	Donc pour vérifier le CPU, on vérifie... Le CPU! Pas la source.
	Donc on raisonne directement sur le """silicium"""
\end{frame}

\section{Educational Aspects (gdd)}

\begin{frame}
	\frametitle{Learning Outcomes}
	testimonials from ORE who became "real" hardware people
	talent pool to recruit from
	MC is an interactive tool
	lecture at Harvard
	16 yo
	graphics rendering
	asm
	LLVM-like optimization stages for MCHPRS

	what we learned
\end{frame}

\begin{frame}[fragile]
	\frametitle{Questions?}
	\begin{columns}
		\begin{column}{0.5\textwidth}
			A big thank you to a few special people
			\begin{itemize}
				\item Hadrien Barral
				\item Rémy Citerin
			\end{itemize}

			Recap

			\begin{itemize}
				\item \hw{} design in MC
				\item Formal verification
			\end{itemize}
		\end{column}
		\begin{column}{0.5\textwidth}
			\begin{figure}
				\caption*{Craft your IC project}
				\qrcode[height=4cm]{\rvlink}
			\end{figure}
		\end{column}
	\end{columns}
	\begin{centering}
		\small \names{}
	\end{centering}
\end{frame}

\newpage
\bibliographystyle{plain}
\bibliography{ref}

\end{document}
